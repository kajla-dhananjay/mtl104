%
% This is the LaTeX template file for lecture notes for CS267,
% Applications of Parallel Computing.  When preparing 
% LaTeX notes for this class, please use this template.
%
% To familiarize yourself with this template, the body contains
% some examples of its use.  Look them over.  Then you can
% run LaTeX on this file.  After you have LaTeXed this file then
% you can look over the result either by printing it out with
% dvips or using xdvi.
%

\documentclass[twoside]{article}
\setlength{\oddsidemargin}{0.25 in}
\setlength{\evensidemargin}{-0.25 in}
\setlength{\topmargin}{-0.6 in}
\setlength{\textwidth}{6.5 in}
\setlength{\textheight}{8.5 in}
\setlength{\headsep}{0.75 in}
\setlength{\parindent}{0 in}
\setlength{\parskip}{0.1 in}

%
% ADD PACKAGES here:
%

\usepackage{amsmath,amsfonts,graphicx,xcolor}

%
% The following commands set up the lecnum (lecture number)
% counter and make various numbering schemes work relative
% to the lecture number.
%
\newcounter{lecnum}
\renewcommand{\thepage}{\thelecnum-\arabic{page}}
\renewcommand{\thesection}{\thelecnum.\arabic{section}}
\renewcommand{\theequation}{\thelecnum.\arabic{equation}}
\renewcommand{\thefigure}{\thelecnum.\arabic{figure}}
\renewcommand{\thetable}{\thelecnum.\arabic{table}}

%
% The following macro is used to generate the header.
%
\newcommand{\lecture}[3]{
	\pagestyle{myheadings}
	\thispagestyle{plain}
	\newpage
	\setcounter{lecnum}{#1}
	\setcounter{page}{1}
	\noindent
	\begin{center}
		\framebox{
			\vbox{\vspace{2mm}
				\hbox to 6.28in { {\bf MTL104 : Linear Algebra
						\hfill Spring 2020-21} }
				\vspace{4mm}
				\hbox to 6.28in { {\Large \hfill Lecture #1  \hfill} }
				\vspace{2mm}
				\hbox to 6.28in { {\ Date : #2 \hfill Scribe: #3} }
				\vspace{2mm}}
		}
	\end{center}
	\markboth{Lecture #1}{Lecture #1}
	
	{\bf Note}: {\it LaTeX template courtesy of UC Berkeley EECS dept.}
	
	{{\bf Disclaimer}: These notes are {\bf unofficial} and were meant for personal use. Therefore accuracy of these notes are not guaranteed. And therefore the liability for any factual errors does not lie with either the author or the instructor.}
}
%
% Convention for citations is authors' initials followed by the year.
% For example, to cite a paper by Leighton and Maggs you would type
% \cite{LM89}, and to cite a paper by Strassen you would type \cite{S69}.
% (To avoid bibliography problems, for now we redefine the \cite command.)
% Also commands that create a suitable format for the reference list.
\renewcommand{\cite}[1]{[#1]}
\def\beginrefs{\begin{list}%
		{[\arabic{equation}]}{\usecounter{equation}
			\setlength{\leftmargin}{2.0truecm}\setlength{\labelsep}{0.4truecm}%
			\setlength{\labelwidth}{1.6truecm}}}
	\def\endrefs{\end{list}}
\def\bibentry#1{\item[\hbox{[#1]}]}

%Use this command for a figure; it puts a figure in wherever you want it.
%usage: \fig{NUMBER}{SPACE-IN-INCHES}{CAPTION}
\newcommand{\fig}[3]{
	\vspace{#2}
	\begin{center}
		Figure \thelecnum.#1:~#3
	\end{center}
}
% Use these for theorems, lemmas, proofs, etc.
\newtheorem{theorem}{Theorem}[lecnum]
\newtheorem{lemma}[theorem]{Lemma}
\newtheorem{proposition}[theorem]{Proposition}
\newtheorem{claim}[theorem]{Claim}
\newtheorem{corollary}[theorem]{Corollary}
\newtheorem{definition}[theorem]{Definition}
\newenvironment{proof}{{\bf Proof:}}{\hfill\rule{2mm}{2mm}}

% **** IF YOU WANT TO DEFINE ADDITIONAL MACROS FOR YOURSELF, PUT THEM HERE:

\newcommand\E{\mathbb{E}}

\begin{document}
	%FILL IN THE RIGHT INFO.
	%\lecture{**LECTURE-NUMBER**}{**DATE**}{**LECTURER**}{**SCRIBE**}
	\lecture{5}{11 Feburary 2021}{Dhananjay Kajla}
	%\footnotetext{These notes are partially based on those of Nigel Mansell.}
	
	% **** YOUR NOTES GO HERE:
	
	% Some general latex examples and examples making use of the
	% macros follow.  
	%**** IN GENERAL, BE BRIEF. LONG SCRIBE NOTES, NO MATTER HOW WELL WRITTEN,
	%**** ARE NEVER READ BY ANYBODY.
	\section{Direct Sum}
	\subsection{Direct sum of subspaces}
	{\bf Theorem : }$V = W_1 \oplus W_2 \iff V = W_1 + W_2$ and $W_1 \cap W_2 = \{0\}$  \\
	
	{\bf Proof($ \implies $)} :
	\begin{itemize}
		\item Suppose $V = W_1 \oplus W_2$
		\item $V = W_1 \oplus W_2 \implies V = W_1 + W_2$
		\item To prove : $W_1 \cap W_2 = \{0\}$ 
		\item Let $z = W_1 \cap W_2$ where $z \neq 0$
		\item Then : $z = {\color{green}\mathbf{0}} + {\color{blue} z} = {\color{green}z} + {\color{blue}\mathbf{0}}$ where blue elements belong to $W_1$ and green elements belong to $W_2$
		\item As z has been expressed in two ways, therefore the representation is not unique.
		\item Therefore z has to be $\mathbf{0}$. Hence proved.
	\end{itemize}

	{\bf Proof($\impliedby$)}:
	\begin{itemize}
		\item Suppose $V = W_1 + W_2$ and $W_1 \cap W_2 = \{0\}$ 
		\item To Prove : $V = W_1 \oplus W_2$
		\item let $z \in V$. 
		\item Let blue represent the fact that the element belongs to $W_1$ and green elements belong to $W_2$
		\item Suppose the following two representations of z exist:
		\begin{itemize}
			\item $z = {\color{blue}x_1 + \color{green}y_1}$ 
			\item $z = {\color{blue}x_2 + \color{green}y_2}$
		\end{itemize}
		\item Then $x_1 + y_1 = x_2 + y_2 = z$
		\item ${\color{blue}x_1 - x_2} = {\color{green}y_1 - y_2}$
		\item Clearly LHS = RHS
		\item But as ${\color{blue}W_1} \cap {\color{green}W_2} = \{0\}$
		\item Therefore, $x_1 - x_2 = 0$ and $y_1 - y_2 = 0$
		\item Therefore $x_1 = x_2$ and $y_1 = y_2$
		\item Therefore each $z \in V$ is uniquely represented as ${\color{blue}x} + {\color{green}y}$
		\item Therefore $V = W_1 \oplus W_2$
	\end{itemize}
	\subsection{Extended direct sum}
	\begin{itemize}
		\item A vector space V(F) is said to be direct sum of it's subspaces $W_1, W_2, ..., W_k$ i.e. $V = W_1 \oplus W_2 \oplus ... \oplus W_k$ if and only if a vector $z \in V$ is uniquely expressible as $z = x_1 + x_2 + ... + x_k$ with $x_i \in W_i$
		\item $V = W_1 \oplus W_2 \oplus W_3 \oplus ... \oplus W_k$ Iff and only if :
		\begin{enumerate}
			\item $V = W_1 + W_2 + ... + W_k$
			\item $W_i \cap (W_1 + W_2 + ... + W_{i-1} + W_{i+1} + ... + W_{k}) = \{0\}$ for each $i=1,2,..,k$
		\end{enumerate}
	\end{itemize}
	\subsection{Complementary Subspaces}
	\begin{itemize}
		\item If $V = W_1 \oplus W_2 (\implies V = W_1 + W_2  \text{ and } W_1 \cap W_2 = \{0\})$
		\item In such a case $W_1 \text{ and } W_2$ are said to be complementary subspaces.
		\item Complementary subspaces are {\bf NOT} unique.
		\item For example : Consider the vector space $\mathbb{R}^{3}(R)$, then the following are it's subspaces :
		\begin{itemize}
			\item $M_1 = \{(0,x_2,x_3)\}$ (yz-plane), $N_1 = \{(x_1,0,0)\}$ (x-axis)
			\item $M_2 = \{(x_1,0,x_3)\}$ (xz-plane), $N_2 = \{(0,x_2,0)\}$ (y-axis)
			\item $M_3 = \{(x_1,x_2,0)\}$ (xy-plane), $N_3 = \{(0,0,x_3)\}$ (z-axis)
			\item $V = M_1 \oplus N_1 = M_2 \oplus N_2 = M_3 \oplus N_3$
		\end{itemize}
		\item {\bf Theorem: } Corresponding to a given sub-space, there will be a unique complementary subspace.
		\item Proof :
		\begin{itemize}
			\item Let $W_1$ be a given subspace and $W_2$ and $W_3$ be it's complementary subspaces so that $V_1 = W_1 \oplus W_2 = W_1 \oplus W_3$
			\item Now let $z = x_1 + x_2$ where $x_1 \in W_1$ and $x_2 \in W_2$
			\item Also let $z = x_1 + x_3$ where $x_1 \in W_1$ and $x_3 \in W_3$
			\item $x_1 + x_2 = x_1 + x_3$
			\item $x_2 = x_3$
			\item Therefore $W_2 = W_3$
		\end{itemize}
	\end{itemize}
	\subsection{Dimension of a subspace}
	\begin{itemize}
		\item Each subspace $W$ of a finitely generated vector space V(F) of dimension $n$ is a finite dimensional subspace with dimension $\leq n$ 
		\[dim(W) \leq dim(V)\]
		\item $dim(W) = dim(U) \iff W = V$
		\item If $W_1$ and $W_2$ are subspaces of a finite dimension vector space V(F), then :
		\[dim(W_1 + W_2) = dim(W_1) + dim(W_2) - dim(W_1 \cap W_2)\]
		\item $V = W_1 \oplus W_2 \implies dim(V) = dim(W_1) + dim(W_2)$ 
	\end{itemize}
	\subsection{Quotient Space}
	\begin{itemize}
		\item Let $W$ be any subspace of a vector space V(F). Let $x$ be any vector in $V$. Then the set 
		\[W + x = \{w+x : w \in W\}\]
		is called a coset of $W$ in $V$ generated by $x$.
		\[ \frac{V}{W} = \{W + x : x \in V\} = \text{ set of all cosets of W in V} \]
		\item $\mathbf{0} \in V \text{ and } W + \mathbf{0} = W$, therefore $W$ is itself a coset of $W$ in $V$
		\item If $x \in W$, then $W + x = W$(why?)
		\item If $W + x$ and $W + y$ are two cosets of $W$ in $V$, then 
		\[W + x = W + y \iff x - y \in W\]
		\item Proof $(\implies)$
		\begin{itemize}
			\item Since $\mathbf{0} \in W \implies \mathbf{0} + x \in W + x$, thus $x \in W + x$
			\item Now $W + x = W + y \implies x \in W + y$
			\item $x - y  \in W + (y-y)$
			\item $x - y \in W + \mathbf{0}$
		\end{itemize}
		\item Proof $(\impliedby)$
		\begin{itemize}
			\item $x - y \in W \implies W + x -y = W$
			\item $W + (x -y + y) = W + y$
			\item $W + x = W + y$
		\end{itemize}
		\item {\bf Theorem :} If $W$ is a subspace of a vector space V(F), then set $\frac{V}{W}$ of all cosets $W + x$ where $x$ is any arbitrary element of $V$, is a vector space over F, under the operations defined by : 
		\begin{itemize}
			\item {\bf Vector Addition : } $\forall x,y \in V \text{, } (W + x) + (W + y) = W + x + y$
			\item {\bf Scalar Multiplication : } $\alpha (W + x) = W + \alpha x$
		\end{itemize}
		\item 
		\[dim\left(\frac{V}{W}\right) = dim(V) - dim(W)\]
		\begin{itemize}
			\item Let $m$ be the dimension of the subspace $W$ of the vector space V(F)
			\item Let $S_1 = \{x_1, x_2, ..., x_m\}$ be a basis of W.
			\item Since $S_1$ is a linearly independent subset of V, it can be extended to form a basis of V.
			\item Let $S_2 = \{x_1, x_2, ..., x_m, y_1, y_2, ..., y_n\}$
			\item $dim(V) = m + n$, $dim(W) = m$
			\item $dim(V) - dim(W) = m + n - m = n$
			\item Therefore We have $dim\left(\frac{V}{W}\right)=n$
			\item Claim : $S_3 = \{W + y_1, W + y_2, ..., W + y_n\}$ is a basis of $\frac{V}{W}$.
			\item Prove that $L(S_3) = \frac{V}{W}$ and that $S_3$ is linearly independent.
			\item To prove linear independence of $S_3$,
			\begin{itemize}
				\item $\alpha_1 (W + y_1) + \alpha_2 (W + y_2) + \alpha_3 (W + y_3) + ... + \alpha_n (W + y_n) = \mathbf{0}_{\frac{V}{W}}$
				\item $W + (\alpha_1 y_1 + \alpha_2 y_2 + \alpha_3 y_3 + ... + \alpha_n y_n) = W + \mathbf{0}$
				\item $\alpha_1 y_1 + \alpha_2 y_2 + ... + \alpha_n y_n \in W$
				\item $\sum_{i=1}^{n} \alpha_i y_i =$ L.C. of $S_1$ (Any vector in $W$ can be expressed as a L.C. of basis vectors.)
				\item $\sum_{i=1}^{n} \alpha_i y_i = \sum_{i=1}^{m} \beta_i x_i$
				\item $\sum_{i=1}^{n} \alpha_i y_i - \sum_{i=1}^{m} \beta_i x_i = 0$
				\item $\alpha_i = 0$ and $\beta_i = 0$
			\end{itemize}
			\item To prove $L(S_3) = \frac{V}{W}$
			\begin{itemize}
				\item Let $W + x \in \frac{V}{W}$
				\item $x \in V \implies x = \sum_{i = 1}^{n}\alpha_i x_i + \sum_{i=1}^{m} \beta_i y_i$
				\item $x = z + \sum_{i = 1}^{m}\beta_i y_i$ where $z =\sum_{i = 1}^{n}\alpha_i x_i$
				\item $W + x = W + (z + \sum_{i = 1}^{m}\beta_i y_i)$
				\item $W + x = W + \sum_{i=1}^{m}\beta_i y_i$
				\item $W + x = (W + \beta_1 y_1) + (W + \beta_2 y_2) + ... + (W + \beta_m y_m)$
				\item $W + x =$ L.C. of $S_3$
				\item Therefore $L(S_3) = \frac{V}{W}$
			\end{itemize}
			\item $S_2$ is a basis of V
			\item To prove $L(S_3) = \frac{V}{W}$
		\end{itemize}
		\item $x, y \in V \implies x + y \in V$
		\item Also $\alpha \in F$ and $x \in V \implies \alpha x \in V$
		\item Therefore $W + (x + y) \in \frac{V}{W}$ and also $W + \alpha x \in \frac{V}{W}$
		\item Thus $\frac{V}{W}$ is closed with respect to addition of cosets and scalar multiplication of cosets.
		\item Zero element of $\frac{V}{W}$ is $W + \mathbf{0} = W$
		\item $W - x$ is the inverse of $W + x$
		\item [H.W.] Prove $\frac{V}{W}$ is a vector space.
	\end{itemize}
\end{document}





