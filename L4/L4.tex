%
% This is the LaTeX template file for lecture notes for CS267,
% Applications of Parallel Computing.  When preparing 
% LaTeX notes for this class, please use this template.
%
% To familiarize yourself with this template, the body contains
% some examples of its use.  Look them over.  Then you can
% run LaTeX on this file.  After you have LaTeXed this file then
% you can look over the result either by printing it out with
% dvips or using xdvi.
%

\documentclass[twoside]{article}
\setlength{\oddsidemargin}{0.25 in}
\setlength{\evensidemargin}{-0.25 in}
\setlength{\topmargin}{-0.6 in}
\setlength{\textwidth}{6.5 in}
\setlength{\textheight}{8.5 in}
\setlength{\headsep}{0.75 in}
\setlength{\parindent}{0 in}
\setlength{\parskip}{0.1 in}

%
% ADD PACKAGES here:
%

\usepackage{amsmath,amsfonts,graphicx,xcolor}

%
% The following commands set up the lecnum (lecture number)
% counter and make various numbering schemes work relative
% to the lecture number.
%
\newcounter{lecnum}
\renewcommand{\thepage}{\thelecnum-\arabic{page}}
\renewcommand{\thesection}{\thelecnum.\arabic{section}}
\renewcommand{\theequation}{\thelecnum.\arabic{equation}}
\renewcommand{\thefigure}{\thelecnum.\arabic{figure}}
\renewcommand{\thetable}{\thelecnum.\arabic{table}}

%
% The following macro is used to generate the header.
%
\newcommand{\lecture}[3]{
	\pagestyle{myheadings}
	\thispagestyle{plain}
	\newpage
	\setcounter{lecnum}{#1}
	\setcounter{page}{1}
	\noindent
	\begin{center}
		\framebox{
			\vbox{\vspace{2mm}
				\hbox to 6.28in { {\bf MTL104 : Linear Algebra
						\hfill Spring 2020-21} }
				\vspace{4mm}
				\hbox to 6.28in { {\Large \hfill Lecture #1  \hfill} }
				\vspace{2mm}
				\hbox to 6.28in { {\ Date : #2 \hfill Scribe: #3} }
				\vspace{2mm}}
		}
	\end{center}
	\markboth{Lecture #1}{Lecture #1}
	
	{\bf Note}: {\it LaTeX template courtesy of UC Berkeley EECS dept.}
	
	{{\bf Disclaimer}: These notes are {\bf unofficial} and were meant for personal use. Therefore accuracy of these notes are not guaranteed. And therefore the liability for any factual errors does not lie with either the author or the instructor.}
}
%
% Convention for citations is authors' initials followed by the year.
% For example, to cite a paper by Leighton and Maggs you would type
% \cite{LM89}, and to cite a paper by Strassen you would type \cite{S69}.
% (To avoid bibliography problems, for now we redefine the \cite command.)
% Also commands that create a suitable format for the reference list.
\renewcommand{\cite}[1]{[#1]}
\def\beginrefs{\begin{list}%
		{[\arabic{equation}]}{\usecounter{equation}
			\setlength{\leftmargin}{2.0truecm}\setlength{\labelsep}{0.4truecm}%
			\setlength{\labelwidth}{1.6truecm}}}
	\def\endrefs{\end{list}}
\def\bibentry#1{\item[\hbox{[#1]}]}

%Use this command for a figure; it puts a figure in wherever you want it.
%usage: \fig{NUMBER}{SPACE-IN-INCHES}{CAPTION}
\newcommand{\fig}[3]{
	\vspace{#2}
	\begin{center}
		Figure \thelecnum.#1:~#3
	\end{center}
}
% Use these for theorems, lemmas, proofs, etc.
\newtheorem{theorem}{Theorem}[lecnum]
\newtheorem{lemma}[theorem]{Lemma}
\newtheorem{proposition}[theorem]{Proposition}
\newtheorem{claim}[theorem]{Claim}
\newtheorem{corollary}[theorem]{Corollary}
\newtheorem{definition}[theorem]{Definition}
\newenvironment{proof}{{\bf Proof:}}{\hfill\rule{2mm}{2mm}}

% **** IF YOU WANT TO DEFINE ADDITIONAL MACROS FOR YOURSELF, PUT THEM HERE:

\newcommand\E{\mathbb{E}}

\begin{document}
	%FILL IN THE RIGHT INFO.
	%\lecture{**LECTURE-NUMBER**}{**DATE**}{**LECTURER**}{**SCRIBE**}
	\lecture{4}{10 Feburary 2021}{Dhananjay Kajla}
	%\footnotetext{These notes are partially based on those of Nigel Mansell.}
	
	% **** YOUR NOTES GO HERE:
	
	% Some general latex examples and examples making use of the
	% macros follow.  
	%**** IN GENERAL, BE BRIEF. LONG SCRIBE NOTES, NO MATTER HOW WELL WRITTEN,
	%**** ARE NEVER READ BY ANYBODY.
	\section{Subspaces}
	\subsection{Examples of subspaces}
	\begin{itemize}
		\item V(F) is also a subspace, so is $\{0\}$.
		\item Let V be the vector space $\mathbb{R}^{3}$. Then the set $W$ consisting of those vectors whose third component is zero, i.e. $w = \{a,b,0 : a,b \in \mathbb{R}\}$ is a subspace of $\mathbb{R}^{3}$
		\item Let V be the vector space of all $n \times n$ matrices. Then the set W consisting of these matrices $A = [a_{ij}]$ for which $a_{ji} = a_{ij}$ (Symmetric matrices) is a subspace of V.
		\item Let V be the vector space of polynomials. Then the set $W$ consisting of polynomials with degree $\leq n$, for a fixed n, is a subspace of V.
		\item Let V be the vector space of all functions for a non-empty set $X$ into the real field $\mathbb{R}$. Then the set consisting of all bounded functions in V is a subspace of V.
		\begin{itemize}
			\item A function $f \in V$ is bounded iff $\exists M \in R$ such that $|f(x)| \leq M$
		\end{itemize}
		\item Let S be a non-empty subset of V(F), The set of all linear comination of vectors in S, denoted by $L(S)$, is a subspace of V containing S.
		\item Furthermore, if $W$ is any other subspace of $V$ containing $S$, then $L(S) \subseteq W$.
		\item The solution space of a system of linear equations $\subseteq R^{n\times 1}$
		\item In $F^{n}$, the set of all n-tuples $(x_1, x_2, ..., x_n)$ with $x_1 = 0$ is a subspace.
		\item The set of all hermitian matrices is {\bf NOT} a subspace of the space of all $n \times n$ matrices over $\mathbb{C}$. The set of all $n \times n$ complex hermitian matrices is a vector space over the field of real numbers.
		\end{itemize}
	\subsection{Few Properties of subspaces}
	\begin{enumerate}
		\item Suppose $W_1$ and $W_2$ are subspaces of a vector space V(F), Then
		\begin{itemize}
			\item $W_1 \cup W_2$ need not be a subspace of V
				\begin{itemize}
					\item For example. Consider the vector space of $\mathbb{R}^{2}$, with $W_1 = \{(a,0); a \in \mathbb{R}\}$ and $W_2 = \{(0,b); b\in \mathbb(R)\}$. Then $(1,0) \in W_1$ and $(0,1) \in W_2$, but their addition $(1,0) + (0,1) = (1,1) \notin W_1 \cup W_2$.
					\item Hence addition is not closed, hence $W_1 \cup W_2$ is not a subspace of V
				\end{itemize}
			\item $W_1 \cap W_2$ is a subspace of V. Proof :  
				\begin{itemize}
					\item $\mathbf{0} \in W_1 \text{, } \mathbf{0} \in W_2 \implies \mathbf{0} \in W_1 \cap W_2$
					\item Let $x, y \in W_1 \cap W_2 \text{, } \alpha \in F$,
					\item Now $x \in W_1 \cap W_2 \implies x \in W_1 \text{ and } x \in W_2$ 
					\item Similarly $y \in W_1 \cap W_2 \implies y \in W_1 \text{ and } y \in W_2$ 
					\item Since $W_1$ is a subspace, $\alpha (x + y) \in W_1$
					\item Similarly $W_2$ is a subspace, so $\alpha (x + y) \in W_2$
					\item Therefor $\alpha (x + y) \in W_1 \cap W_2$
				\end{itemize}
		\end{itemize}
		\item Let $V$ be a vector space over the field F, then intersection of any collection of subspaces (i.e. Arbitrary intersection of subspaces) is a subspace in V. 
	\end{enumerate}
	\subsection{Linear Sum of Subspaces}
	\begin{itemize}
		\item Let $W_1$ and $W_2$ be two subspaces of a vector space V(F), Then the linear sum of subspaces, denoted by $W_1 + W_2$ is defined as :
		\[W_1 + W_2 = \{x_1 + x_2 : x_1 \in W_1, x_2 \in W_2\}\]
		\item {\bf Theorem :} If $W_1$ and $W_2$ are subspaces of a vector space V(F), then $W_1 + W_2$ is a subspace of V(F)
		\item Proof :
			\begin{itemize}
				\item $\mathbf{0} = {\color{green} \mathbf{0}} + {\color{blue} \mathbf{0}}$ where ${\color{green} \mathbf{0}} \in W_1$ and ${\color{blue}\mathbf{0}} \in W_2$
				\item let $x, y \in W_1 + W_2 \text{, } \alpha \in F$, then :
					\begin{itemize}
						\item $x = x_1 + x_2$ where $x_1 \in W_1$ and $x_2 \in W_2$
						\item $y = y_1 + y_2$ where $y_1 \in W_1$ and $y_2 \in W_2$
					\end{itemize}
				\item Consider $\alpha x + y = \alpha (x_1 + x_2) + (y_1 + y_2) = (\alpha x_1 + y_1) + (\alpha x_2 + y_2)$
					\begin{itemize}
						\item Now $(\alpha x_1 + y_1) \in W_1$ as $W_1$ is a subspace 
						\item Similarly $(\alpha x_2 + y_2) \in W_2$ as $W_2$ is a subspace
						\item Therefore $\alpha x + y \in W_1 + W_2$ 
					\end{itemize}
				\item Therefore $W_1 + W_2$ is a subspace of V
			\end{itemize}
		\item If $W_1, W_2, ..., W_k$ are subspaces of the vector space V(F). Then their linear sum, denote by $W_1 + W_2 + ... + W_k$ is defined as :
		\[W_1 + W_2 + ... + W_k = {x_1 + x_2 + ... + x_k : x_i \in W_i, 1 \leq i \leq k}\]
		\item $W_1 + W_2 + ... + W_k$ is a subspace of V(F)
	\end{itemize}
	\subsection{Direct Sum of Subspaces}
	\begin{itemize}
		\item The vector space V over the field F is the direct sum of two vector spaces $W_1$ and $W_2$ if 
			\begin{enumerate}
				\item $V = W_1 + W_2$
				\item Every vector $z\ in V$ can be uniquely expressed as the sum of $x_1 + x_2$ where $x_1 \in W_1$ and $x_2 \in W_2$
			\end{enumerate}
		\item If the vector space V(F) is the direct sum of two subspaces, $W_1$ and $W_2$, we write,
		\[V = W_1 \oplus W_2\]
		\item {\bf Theorem :} $V = W_1 \oplus W_2 \iff V = W_1 + W_2$ and $W_1 \cap W_2 = \{0\}$
	\end{itemize}
\end{document}





